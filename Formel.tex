\documentclass[11pt]{article}

\usepackage{amsmath}
\usepackage{amssymb}
\usepackage[linguistics]{forest}
\usepackage{mathtools}
\usepackage{bbm}
\usepackage[utf8]{inputenc} % this is needed for umlauts
\usepackage[ngerman]{babel} % this is needed for umlauts
\usepackage[T1]{fontenc}
\usepackage{fancyhdr}
\usepackage[makeroom]{cancel}
\usepackage{tikz}
\usepackage{enumitem}
\usepackage{ulem}
\usepackage{graphicx}
\usepackage{bm}

\graphicspath{{images/}}
\setlength{\parindent}{0pt}

\expandafter\def\expandafter\normalsize\expandafter{%
    \normalsize
    \setlength\abovedisplayskip{20pt}
    \setlength\belowdisplayskip{15pt}
    \setlength\abovedisplayshortskip{15pt}
    \setlength\belowdisplayshortskip{13pt}
}

\input ulem.sty

\newlist{arrowlist}{itemize}{1}
\setlist[arrowlist]{label=$\Rightarrow$}

\tikzset{node distance = 1cm and 5cm}
\usetikzlibrary{trees}
\usepackage[a4paper,bindingoffset=0.2in,
            left=1in,right=1in,top=1.2in,bottom=1in,
            footskip=.25in]{geometry}

\pagestyle{fancy}
\fancyhf{}
\rfoot{Page \thepage}
\DeclareMathOperator*{\argmin}{arg\,min}
\DeclareMathOperator*{\argmax}{arg\,max}
\DeclarePairedDelimiter\ceil{\lceil}{\rceil}
\DeclarePairedDelimiter\floor{\lfloor}{\rfloor}
\DeclarePairedDelimiter\abs{\lvert}{\rvert}
\DeclarePairedDelimiter\norm{\lVert}{\rVert}
\DeclareMathOperator{\rank}{rank}


% Custom colors
\usepackage{color}
\definecolor{deepblue}{rgb}{0,0,0.5}
\definecolor{deepred}{rgb}{0.6,0,0}
\definecolor{deepgreen}{rgb}{0,0.5,0}

\usepackage{listings}

% Python style for highlighting
\newcommand\pythonstyle{\lstset{
language=Python,
basicstyle=\ttm,
otherkeywords={self},             % Add keywords here
keywordstyle=\ttb\color{deepblue},
emph={MyClass,__init__},          % Custom highlighting
emphstyle=\ttb\color{deepred},    % Custom highlighting style
stringstyle=\color{deepgreen},
frame=tb,                         % Any extra options here
showstringspaces=false            %
}}


% Python environment
\lstnewenvironment{python}[1][]
{
\pythonstyle
\lstset{#1}
}
{}

% Python for external files
\newcommand\pythonexternal[2][]{{
\pythonstyle
\lstinputlisting[#1]{#2}}}

% Python for inline
\newcommand\pythoninline[1]{{\pythonstyle\lstinline!#1!}}

\begin{document}
\begin{equation}
  E=\frac{J(J+1)\hbar^2}{2I}
\end{equation}
\begin{equation}
  I=\frac{m_1\cdot m_2}{m_1+m_2}{r^2}
\end{equation}
\begin{equation}
  E=\hbar w(n+\frac{1}{2})
\end{equation}
\begin{equation}
  \nu=\frac{\omega}{2\pi}=\sqrt{\frac{k}{\mu}}
\end{equation}
\begin{equation}
  E_{ges}=\left(n+\frac{1}{2}\right)h\nu+\frac{\hbar^2J(J+1)}{2I}
\end{equation}
\begin{equation}
  P=\frac{N\cdot V_{\text{Atom}}}{V_{\text{Elementarzelle}}}
\end{equation}
\begin{equation}
  2d_{hkl}\sin{\theta}=\lambda
\end{equation}
\begin{equation}
  v_{ph}=\frac{\omega}{q}
\end{equation}
\begin{equation}
  v_g=\frac{\partial w}{\partial q}
\end{equation}
\begin{equation}
  \bm{G}=\frac{2\pi}{a}
\end{equation}
Dispersionsrelation lineare Kette:
\begin{equation}
  \omega=2\sqrt{\frac{C_1}{M}}\left|\sin\left(\frac{qa}{2}\right)\right|
\end{equation}
Lösungsansatz:
\begin{equation}
  u_{s+n}=Ue^{-i[\omega t-q(s+n)a]}
\end{equation}
Modifizierte Streubedingung:
\begin{equation}
  \begin{align}
    \hbar\omega&=\hbar\omega_0+\hbar\omega_{\bm{q}}\\
    \hbar\bm{k}&=\hbar\bm{k}_0\pm\hbar\bm{q}+\hbar\bm{G}
  \end{align}
\end{equation}
Zustandsdichte im reziproken Raum:
\begin{equation}
  \rho_q^{(n)}=\frac{L^{n}}{(2\pi)^n}
\end{equation}
Debye-Näherung $\omega=vq$:
\begin{equation}
  \mathcal{D}(\omega)d\omega=\rho_q\int_\omega^{\omega+dw}d^3q=\rho_qd\omega
  \int_{\omega=const}\frac{dS_\omega}{v_g}=\rho_qd\omega\frac{4\pi q^2}{v_g}=
  \frac{V}{2\pi^2}\frac{\omega^2}{v^3}d\omega
\end{equation}
\begin{equation}
  U=\int_0^{\omega_D}\hbar\omega\mathcal{D}(\omega)\langle n(\omega,T)\rangle
  d\omega
\end{equation}
\begin{equation}
  \begin{align}
    k_B\Theta&=\hbar\omega_D \\
    x&=\frac{\hbar\omega}{k_BT}\\
    x_D&=\frac{\hbar\omega_D}{k_BT}=\frac{\Theta}{T}
  \end{align}
\end{equation}
Immer noch Debye-Näherung:
\begin{equation}
  C_V=\left(\frac{\partial U}{\partial T}\right)_V=
  9Nk_B\left(\frac{T}{\Theta}\right)^3\int_0^{x_D}\frac{x^4e^x}{(e^x-1)^2}dx
\end{equation}
Wärmetransport durch Gitterstöße:
\begin{equation}
  \bm{j}=-\Lambda\nabla T
\end{equation}
\begin{equation}
  \Lambda=\frac{1}{3}Cvl
\end{equation}
Bei tiefen Temperaturen:
\begin{equation}
  \Lambda\propto T^3d
\end{equation}
\begin{equation}
  \psi(\bm{r})=\frac{1}{\sqrt{V}}e^{i\bm{k}{r}}
\end{equation}
\begin{equation}
  E=\frac{\hbar^2k^2}{2m}
\end{equation}
\begin{equation}
  k_i=\frac{2\pi}{L}m_i
\end{equation}
\begin{equation}
  \rho_k=\frac{2V}{(2\pi^3)}
\end{equation}
\begin{equation}
  \mathcal{D}(E)=\frac{2V}{(2\pi^3)\hbar}\frac{4\pi k^2}{v_g}\overset{
  v_g=\frac{\hbar k}{m}}{=}
  \frac{V}{2\pi^2}\left( \frac{2m}{\hbar^2} \right)^{\frac{3}{2}}\sqrt{E}
\end{equation}
Zweidimensionale Zustandsdichte pro Volumen:
\begin{equation}
  D^{(2)}(E)=\frac{\rho_k^{(2)}}{A\hbar}\frac{2\pi k}{v_g}=\frac{m}{\pi\hbar^2}
\end{equation}
\begin{equation}
  f(E)=\frac{1}{e^{\frac{(E-\mu)}{k_BT}}+1}
\end{equation}
\begin{equation}
  n=\frac{N}{V}=\int_0^\infty D(E)f(E,T=0)dE=\int_0^{E_F}D(E)dE=\frac{1}{2\pi^2}
  \left( \frac{2m}{\hbar^2}\right)^{\frac{3}{2}}\frac{2E_F^{\frac{3}{2}}}{3}
\end{equation}
\begin{equation}
  \begin{align}
    E_F&=\frac{\hbar^2}{2m}(3\pi^2n)^{\frac{2}{3}} \\
    k_F&=(3\pi^2n)^{\frac{1}{3}}\quad &\text{Fermi-Wellenvektor} \\
    v_F&=\frac{\hbar}{m}(3\pi^2n)^{\frac{1}{3}}\quad &\text{Fermi-Geschwindigkeit} \\
    T_F&=\frac{E_F}{k_B}\quad &\text{Fermi-Temperatur}
  \end{align}
\end{equation}
\begin{equation}
  u_0=\int_0^\infty ED(E)f(E,T=0)dE=\int_0^{E_F}ED(E)dE=\frac{3n}{5}E_F=
  \frac{3n}{5}k_BT_F
\end{equation}
\begin{equation}
  \delta u(T)=u(T)-u_0=nk_BT\cdot\frac{T}{T_F}
\end{equation}
\begin{equation}
  c_V^{el}=\left(\frac{\partial u}{\partial T}\right)_V\approx\frac{2nk_BT}{T_F}
  \approx\gamma T
\end{equation}
\begin{equation}
  c_V^{ges}=\gamma T + \begin{cases}
  3n_Ak_B & \text{für } T>\Theta \\
  \beta T^3 & \text{für } T\ll\Theta
  \end{cases}
\end{equation}
Dispersionsrelation quasi-freie Elektronen:
\begin{equation}
  E_{\bm{K}}=\frac{\hbar^2k^2}{2m}=E_{\bm{k}+\bm{G}}=\frac{\hbar^2}{2m}
  \abs{\bm{k}+\bm{G}}^2
\end{equation}
\begin{equation}
  \frac{d\bm{v}}{dt}=\frac{1}{\hbar}\frac{d}{dt}\left(\frac{\partial E(\bm{k})}
  {\partial\bm{k}}\right)=\frac{1}{\hbar}\frac{\partial^2E(\bm{k})}
  {\partial\bm{k}\partial\bm{k}}\frac{\partial\bm{k}}{dt}=
  \frac{1}{\hbar^2}\frac{\partial^2E(\bm{k})}
  {\partial\bm{k}\partial\bm{k}}\bm{F}
\end{equation}
\begin{equation}
  \left(\frac{1}{m^*}\right)_{ij}=\frac{1}{\hbar^2}\frac{\partial^2E(\bm{k})}
  {\partial k_i\partial k_j}
\end{equation}
Bloch-Oszillationen:
\begin{equation}
  \abs{\bm{v}}=\frac{e\mathcal{E}}{\hbar}
\end{equation}
\begin{equation}
  T_B=\frac{\frac{2\pi}{a}}{\frac{e\mathcal{E}}{\hbar}}=\frac{h}{ae\mathcal{E}}
\end{equation}
Drude Modell: Bewegung Elektronen $\iff$ Kinetische Gastheorie
\begin{equation}
  m\frac{d\bm{v}}{dt}=-e\mathcal{E}-m\frac{\bm{v}_d}{\tau}
\end{equation}
\begin{equation}
  \bm{v}_d=-\frac{e\tau}{m}\bm{\mathcal{E}}=-\mu\bm{\mathcal{E}}
\end{equation}
\begin{equation}
  \bm{j}=-en\bm{v}_d=\frac{ne^2\tau}{m}\bm{\mathcal{E}}=ne\mu\bm{\mathcal{E}}
\end{equation}
\begin{equation}
  \sigma=\frac{j}{\mathcal{E}}=\frac{ne^2\tau}{m}=ne\mu
\end{equation}
\begin{equation}
  l=v_F\tau
\end{equation}
\begin{equation}
  \rho=\frac{1}{\sigma}=\frac{m}{ne^2\tau}=\rho_D+\rho_G=\frac{m}{ne^2\tau_D}+
  \frac{m}{ne^2\tau_G(T)}
\end{equation}
\begin{equation}
  \frac{\rho(300K)}{\rho(4.2K)}
\end{equation}
\begin{equation}
  B_C(T)=B_C(0)\left[1-\left(\frac{T}{T_C}\right)^2\right]
\end{equation}
\begin{equation}
  \sigma=e(n\mu_n+p\mu_p)
\end{equation}
\begin{equation}
  \begin{align}
    n&=\int_{E_L}^{\infty}D_L(E)f(E,T)dE\\
    p&=\int_{-\infty}^{E_V}D_V(E)[1-f(E,T)]dE
  \end{align}
\end{equation}
\begin{equation}
  \begin{align}
    D_L(E)&=\frac{1}{2\pi^2}\left(\frac{2m_n^*}{\hbar^2}\right)^{\frac{3}{2}}
    \sqrt{E-E_L}\\
    D_V(E)&=\frac{1}{2\pi^2}\left(\frac{2m_p^*}{\hbar^2}\right)^{\frac{3}{2}}
    \sqrt{E_V-E}
  \end{align}
\end{equation}
\begin{equation}
  n=\int D_L(E)f(E)dE=\frac{1}{2\pi^2}\left(\frac{2m_n^*}{\hbar^2}\right)^
  {\frac{3}{2}}e^{\frac{E_F}{k_BT}}\int_{E_L}^{\infty}\sqrt{E-E_L}
  e^{\frac{E}{k_BT}}
\end{equation}
Massenwirkung:
\begin{equation}\label{massenwirkung}
  \begin{align}
    n=&\mathcal{N}_Le^{-\frac{(E_L-E_F)}{k_BT}}\\
    p=&\mathcal{N}_Ve^{\frac{(E_V-E_F)}{k_BT}}
  \end{align}
\end{equation}
Für intrinsische Halbleiter:
\begin{equation}
  n\cdot p=\mathcal{N}_L\mathcal{N}_Ve^{-\frac{E_g}{k_BT}}
\end{equation}
\begin{equation}
  n_i=p_i=\sqrt{\mathcal{N}_L\mathcal{N}_V}e^{-\frac{E_g}{2k_BT}}
\end{equation}
\begin{equation}
  E_F=\frac{E_L+E_V}{2}+\frac{k_BT}{2}\ln(\frac{\mathcal{N}_V}{\mathcal{N}_L})
  =\frac{E_L+E_V}{2}+\frac{3}{4}k_BT\ln(\frac{m^*_p}{m^*_n})
\end{equation}
Dotierte Halbleiter:






\end{document}
